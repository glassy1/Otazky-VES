\documentclass[a4paper, 12pt]{article}
    \usepackage{baskervillef}
    %\usepackage{fouriernc}
    \usepackage[T1]{fontenc}
    \usepackage[utf8]{inputenc}
    \usepackage[czech]{babel}
    \usepackage{amsmath}
\title{Otázky k procvičování znalostí z předmětu KEV/VES}
\author{Ing. Tomáš Glasberger, Ph.D.}
\begin{document}
\maketitle

\section{Přednáška č. 1}
\begin{enumerate}
    \item Jak je definován celkový účiník?
    \item Jaký je vztah pro činný výkon základní harmonické?
    \item Jaký je vztah pro zdánlivý výkon základní harmonické?
    \item Vyjádřete efektivní hodnotu proudu jako součet harmonických složek.
    \item Odvoďte vztah pro celkový zdánlivý výkon.
    \item Na jaké složky lze celkový zdánlivý výkon rozdělit z hlediska harmonických proudu?
    \item Nakreslete kruhový diagram výkonů jednoho usměrňovače --- popište osy.
    \item Nakreslete blokové schéma dvou sériové řazených usměrňovačů. 
    \item Jak se rozdělí usměrněné napětí vzhledem k celkovému výstupnímu napětí?
    \item Jak se rozdělí činný a jalový výkon?
    \item Vysvětlete princip postupného řízení usměrňovačů za účelem snížení
      odebíraného jalového výkonu.
    \item Nakreslete kruhový diagram výkonů pro toto zapojení měničů.
    \item Nakreslete tvar proudu odebíraný můstkovým usměrňovačem při práci do
        $RL$ zátěže s předpokladem $L\rightarrow \infty $.
    \item Proveďte harmonickou analýzu daného průběhu. 
    \item Které harmonické se zde
        vyskytují?
    \item Nakreslete schéma zapojení usměrňovačů pro zmenšování deformačního
      výkonu ze sítě odebíraného proudu.
    \item Jaké má vhodný transformátor hodinové číslo?
    \item Jaké se předpokládají počty závitů jednotlivých vinutí?
    \item Vyjádtřete okamžitou hodnotu proudu odebíraného ze sítě v závislosti
      na proudech v sekundárech transformátoru.
    \item Odvoďte průběhy proudu jednotlivými vinutími a nakreslete je.
    \item Zakreslete proud odebíraný ze sítě.
    \item Jaké tento proud obsahuje (nebo neobsahuje) harmonické?

\end{enumerate}

\section{Přednáška č. 2}
\begin{enumerate}
  \item Schéma paralelního spojení usměrňovačů.
  \end{enumerate}
\end{document}


\documentclass[a4paper, 12pt]{article}
    \usepackage{baskervillef}
    %\usepackage{fouriernc}
    \usepackage[T1]{fontenc}
    \usepackage[utf8]{inputenc}
    \usepackage[czech]{babel}
\title{Otázky k procvičování znalostí z předmětu KEV/VES}
\author{Ing. Tomáš Glasberger, Ph.D.}
\begin{document}
\maketitle
\section{Přednáška č. 1}
\begin{enumerate}
    \item Jak je definován celkový účiník?
    \item Jaký je vztah pro činný výkon základní harmonické?
    \item Jaký je vztah pro zdánlivý výkon základní harmonické?
    \item Vyjádřete efektivní hodnotu proudu jako součet harmonických složek.
    \item Odvoďte vztah pro celkový zdánlivý výkon.
    \item Na jaké složky lze celkový zdánlivý výkon rozdělit z hlediska harmonických proudu?
    \item Nakreslete kruhový diagram jednoho usměrňovače --- popište osy.
    \item Nakreslete blokové schéma dvou sériové řazených usměrňovačů. 
    \item Jak se rozdělí usměrněné napětí vzhledem k celkovému výstupnímu napětí?
    \item 
\end{enumerate}
\end{document}

